%% ----------------------------------------------------------------
%% GDP.tex
%% ---------------------------------------------------------------- 
\documentclass{desex3}         % Use the GDP Report Style
\usepackage{multicol}
\usepackage{graphicx}
\usepackage{epstopdf}
\usepackage{caption}
\usepackage{subcaption}
\usepackage[nodayofweek]{datetime}
%\usepackage{lipsum}% dummy text
%\usepackage{todonotes}
%\usepackage[disable]{todonotes} %uncomment this to disable to do notes
\newcommand{\inote}[1] {\todo[inline]{#1}}
%% \removecolourlinks    % Uncomment this command to remove colour from any links
\newcommand{\cellname}[1] {\clearpage \subsection{#1} \makebox[\linewidth]{\rule{\textwidth}{0.4pt}}}
\newcommand{\designer}[1] {{\bf Designer: } #1\\}
\newcommand{\celldescription}[1]{{\bf Cell Description: } #1\\}

%\newcommand{\topimages}[3]{
%\begin{multicols}{3}
%\subsection*{Symbol}
%\includegraphics[width=0.5\textwidth,height=4cm,keepaspectratio=true]{#1}
%\subsection*{Dimensions}
%\includegraphics[width=0.5\textwidth,height=4cm,keepaspectratio=true]{#3}
%\end{multicols}
%\subsection*{Circuit Diagram}
%\includegraphics[width=\textwidth,height=3cm,keepaspectratio=true]{#2}
%}

\newcommand{\topimages}[3]{
\begin{figure}[ht!]
        \centering
        \begin{subfigure}[b]{0.5\textwidth}
			\subsection*{Symbol}
                \includegraphics[width=\textwidth,height=5cm,keepaspectratio]{#1}
        \end{subfigure}%
        ~ %add desired spacing between images, e. g. ~, \quad, \qquad etc.
          %(or a blank line to force the subfigure onto a new line)
        \begin{subfigure}[b]{0.5\textwidth}
				\subsection*{Dimensions}
                \includegraphics[width=\textwidth,height=5cm,keepaspectratio]{#3}
        \end{subfigure}
		~ %add desired spacing between images, e. g. ~, \quad, \qquad etc.
          %(or a blank line to force the subfigure onto a new line)
        \begin{subfigure}[b]{0.7\textwidth}
				\subsection*{Gate Level Diagram}
                \includegraphics[width=\textwidth]{#2}
        \end{subfigure}
\end{figure}
}
%\newcommand{\topimagesgate}[2]{\begin{multicols}{2}\subsection*{Symbol}\includegraphics[width=\textwidth,height=4cm,keepaspectratio=true]{#1}\subsection*{Dimensions}\includegraphics[width=\textwidth,height=4cm,keepaspectratio=true]{#2}\end{multicols}}

\newcommand{\topimagesgate}[2]{
\begin{figure}[ht!]
        \centering
        \begin{subfigure}[b]{0.5\textwidth}
			\subsection*{Symbol}
                \includegraphics[width=\textwidth]{#1}
        \end{subfigure}%
        ~ %add desired spacing between images, e. g. ~, \quad, \qquad etc.
          %(or a blank line to force the subfigure onto a new line)
        \begin{subfigure}[b]{0.5\textwidth}
				\subsection*{Dimensions}
                \includegraphics[width=\textwidth]{#2}
        \end{subfigure}
\end{figure}
}

\newcommand{\sysverilog}[1]{\subsection*{System Verilog Simulation} \begin{figure}[h!] \centering \includegraphics[width=\textwidth]{#1} \end{figure}}

\newcommand{\acchar}[1]{\subsection*{AC Characteristics}\begin{table}[h!]\centering\begin{tabular}{cc}Signal & Average Delay (ps) \\ \hline\input{#1} \end{tabular}\end{table}}


\addtolength{\topmargin}{-.9in} % to help everything fit on one page
\addtolength{\textheight}{2in}

\newcommand{\nosection}[1]{%
  \refstepcounter{section}%
  \addcontentsline{toc}{section}{\protect\numberline{\thesection}#1}%
  \markright{#1}}
	
%% ----------------------------------------------------------------
\begin{document}

\title{Cell Library Databook}

\author{by Team S5\\ \vspace{4cm} \\ H. Lovett (hl13g10) \\ A. J. Robinson (ajr2g10) \\ C. Schepens (cs7g10) \\ M. Wearn (mw20g10)\vspace{4cm}}
%\frontmatter
\maketitle
\cleardoublepage
\tableofcontents
\renewcommand*{\thesection}{\roman{section} }
\newpage
\section*{Introduction}

The set of cells contained within this library are based upon $0.35\mu m$ unified CMOS technology. The follow a two-layer metal design on top of a p-type substrate with N-Well and P-Well regions for pull-up and pull-down networks respectively. All cells are a common $17.2\mu m$ high, with varied widths all integer multiples of 1.2µm. 
All transistors are fixed at $W_P = 2.4\mu m$, $L_P = 0.35\mu m$, $W_N = 1.5\mu m$, $L_N = 0.35\mu m$.


Global signals and power rails are arranged horizontally in \textit{metal1}.
Figure \ref{fig:globalsignals} shows this arrangement with the dimensions of the global signals.
While cell I/O signals are arranged vertically in \textit{metal2} and aligned to a $1.2\mu m$ grid.
Power rails are $1.25\mu m$ wide, while other horizontal signals are $0.5\mu m$ wide. 
The distance between horizontal signals is $0.8\mu m$ from centre to centre. 
Both rails are formed using a continuous ohmic region and line of taps. 

\begin{figure}[htb!]
\includegraphics[width=\textwidth]{cellglobals.pdf}
\caption{Global signals common to all cells}
\label{fig:globalsignals}
\end{figure}

Vertical signals lines are $0.6\mu m$ wide with position of each signal measured from the left edge of the cell to the right edge of the metal strip and detailed on each cell page. 

AC characteristics of cells are measured as the propagation delay from each input to each output under normal operation conditions. 
This is set as having each input driven through one inverter and each output loaded by two inverters from this library. 
Each cell lists both the delay to correct output as a result of an input going high, as well as an input going low. 

Major cells in the library have additional sections detailing the stick diagram and transistor layout of each cell as designed by the designated team member. 
Even though only 4 members were in the group, both the half adder and XOR2 have been included in the library and detailed. 

\setcounter{section}{0}
%% ----------------------------------------------------------------
%\mainmatter
\renewcommand*{\thesection}{\arabic{section} }

\cleardoublepage
    \vspace*{\fill}
    \begin{center}
		\nosection{Cell Library Databook}
      {\Huge CELL LIBRARY DATABOOK }
    \end{center}
    \vspace*{\fill}
\cleardoublepage


\cellname{And2}
\designer{Constantijn Schepens}
\celldescription{A two input AND gate}
\topimages{../and2/symbol.pdf}{../and2/circuitdiagram.jpg}{../and2/blackbox.pdf}

\acchar{../and2/acresults.txt}

\sysverilog{../and2/sv.jpg}


\cellname{buffer}
\designer{Ashley Robinson}
\celldescription{A non-inverting buffer}
\topimages{../buffer/symbol.jpg}{../buffer/circuitdiagram.jpg}{../buffer/blackbox.jpg}

\acchar{../buffer/acresults.txt}

\sysverilog{../buffer/sv.jpg}

\cellname{FULLADDER}
\designer{Martin Wearn}
\celldescription{Adds two bit values and the previous bitsclie carry out, to produce a sum and carry}
\topimages{../fulladder/symbol.png}{../fulladder/circuitdiagram.pdf}{../fulladder/blackbox.pdf}

\acchar{../fulladder/acresults.txt}

\sysverilog{../fulladder/sv.pdf}

\cellname{halfadder}
\designer{Martin Wearn}
\celldescription{Adds two bits to produce a sum and carry}
\topimages{../halfadder/symbol.png}{../halfadder/circuitdiagram.png}{../halfadder/blackbox.pdf}

\acchar{../halfadder/acresults.txt}

\sysverilog{../halfadder/sv.pdf}

\cellname{Inverter}
\designer{Henry Lovett}
\celldescription{A basic inverter gate}
\topimages{../inv/symbol.jpg}{../inv/circuitdiagram.jpg}{../inv/blackbox.jpg}

\acchar{../inv/acresults.txt}

\sysverilog{../inv/sv.jpg}

\cellname{LEFTBUF}
\designer{Henry Lovett}
\celldescription{A start of row buffer cell.}
\topimages{../leftbuf/symbol.pdf}{../leftbuf/circuitdiagram.pdf}{../leftbuf/blackbox.pdf}

\acchar{../leftbuf/acresults.txt}

\sysverilog{../leftbuf/sv.pdf}

\cellname{mux2}
\designer{Constantijn Schepens}
\celldescription{A two input Multiplexor}
\topimages{../mux2/symbol.jpg}{../mux2/circuitdiagram.jpg}{../mux2/blackbox.jpg}

\acchar{../mux2/acresults.txt}

\sysverilog{../mux2/sv.jpg}

\cellname{nand2}
\designer{Constantijn Schepens}
\celldescription{A two input NAND gate}
\topimages{../nand2/symbol.jpg}{../nand2/circuitdiagram.jpg}{../nand2/blackbox.jpg}

\acchar{../nand2/acresults.txt}

\sysverilog{../nand2/sv.jpg}

\cellname{NAND3}
\designer{Constantijn Schepens}
\celldescription{A three input NAND gate}
\topimagesgate{../nand3/symbol.pdf}{../nand3/blackbox.pdf}

\acchar{../nand3/acresults.txt}

\sysverilog{../nand3/sv.pdf}


\cellname{nand4}
\designer{Constantijn Schepens}
\celldescription{A two input NAND gate}
\topimages{../nand4/symbol.jpg}{../nand4/circuitdiagram.jpg}{../nand4/blackbox.jpg}

\acchar{../nand4/acresults.txt}

\sysverilog{../nand4/sv.jpg}

\cellname{nor2}
\designer{Henry Lovett}
\celldescription{A two input NOR gate}
\topimagesgate{../nor2/symbol.pdf}{../nor2/blackbox.pdf}

\acchar{../nor2/acresults.txt}

\sysverilog{../nor2/sv.pdf}

\cellname{nor3}
\designer{Henry Lovett}
\celldescription{A three input NOR gate}
\topimagesgate{../nor3/symbol.pdf}{../nor3/blackbox.pdf}

\acchar{../nor3/acresults.txt}

\sysverilog{../nor3/sv.pdf}

\cellname{OR2}
\designer{Henry Lovett}
\celldescription{A two input OR gate}
\topimages{../or2/symbol.pdf}{../or2/gates.pdf}{../or2/blackbox.pdf}

\acchar{../or2/acresults.txt}

\sysverilog{../or2/sv.pdf}
%\input{../rdtype/rdtype.tex}

\cellname{rightend}
\designer{Henry Lovett}
\celldescription{An end of row buffer cell.}
\topimagesgate{../rightend/symbol.pdf}{../rightend/blackbox.pdf}

\acchar{../rightend/acresults.txt}

\sysverilog{../rightend/sv.jpg}
%
\cellname{rowcrosser}
\designer{Martin Wearn}
\celldescription{A rowcrossing cell}
\topimages{../rowcrosser/symbol.jpg}{../rowcrosser/circuitdiagram.jpg}{../rowcrosser/blackbox.jpg}

\acchar{../rowcrosser/acresults.txt}

\sysverilog{../rowcrosser/sv.jpg}

\cellname{scandtype}
\designer{Constantijn Schepens}
\celldescription{A Raw DType cell}
\topimages{../scandtype/symbol.jpg}{../scandtype/circuitdiagram.jpg}{../scandtype/blackbox.pdf}

\acchar{../scandtype/acresults.txt}

\sysverilog{../scandtype/sv.jpg}

\cellname{scanreg}
\designer{Constantijn Schepens}
\celldescription{A scannable register cell}
\topimages{../scanreg/symbol.jpg}{../scanreg/circuitdiagram.jpg}{../scanreg/blackbox.pdf}

\acchar{../scanreg/acresults.txt}

\sysverilog{../scanreg/sv.pdf}

%\input{../smux2/smux2.tex}
%\input{../smux3/smux3.tex}
%
\cellname{tiehigh}
\designer{Martin Wearn}
\celldescription{A tie to Vdd cell}
\topimages{../tiehigh/symbol.jpg}{../tiehigh/circuitdiagram.jpg}{../tiehigh/blackbox.pdf}

\acchar{../tiehigh/acresults.txt}

\sysverilog{../tiehigh/sv.jpg}
%
\cellname{tielow}
\designer{Martin Wearn}
\celldescription{A tie to GND cell}
%\topimages{../tielow/symbol.jpg}{../tielow/circuitdiagram.jpg}{../tielow/blackbox.pdf}

%\acchar{../tielow/acresults.txt}

%\sysverilog{../tielow/sv.jpg}

\newpage

\begin{multicols}{2}
\section{TIE HIGH}
\makebox[\linewidth]{\rule{0.5\textwidth}{0.4pt}}
{\bf Designer: } Martin Wearn\\
{\bf Cell Description: } A tie to Vdd cell.\\
\subsection*{Dimensions}\includegraphics[width=\textwidth,height=4cm,keepaspectratio=true]{../tiehigh/blackbox.pdf}

\section{TIE LOW}
\makebox[\linewidth]{\rule{0.5\textwidth}{0.4pt}} 
{\bf Designer: } Martin Wearn\\
{\bf Cell Description: } A  tie to GND cell.\\
\subsection*{Dimensions}\includegraphics[width=\textwidth,height=4cm,keepaspectratio=true]{../tielow/blackbox.pdf}
\end{multicols}

\section{ROWCROSSER} \makebox[\linewidth]{\rule{\textwidth}{0.4pt}}
{\bf Designer: } Martin Wearn\\
{\bf Cell Description: } A row crossing cell.\\
\subsection*{Dimensions}\includegraphics[width=\textwidth,height=4cm,keepaspectratio=true]{../rowcrosser/blackbox.pdf}





\cellname{trisbuf}
\designer{Ashley Robinson}
\celldescription{A tristate buffer}
\topimages{../trisbuf/symbol.jpg}{../trisbuf/circuitdiagram.jpg}{../trisbuf/blackbox.jpg}

\acchar{../trisbuf/acresults.txt}

\sysverilog{../trisbuf/sv.jpg}

\cellname{xor2}
\designer{Ashley Robinson}
\celldescription{A two input xor gate}
\topimages{../xor2/symbol.jpg}{../xor2/circuitdiagram.jpg}{../xor2/blackbox.jpg}

\acchar{../xor2/acresults.txt}

\sysverilog{../xor2/sv.jpg}

\appendix
\clearpage
\section{Appendix A}

\subsection{Team Management}

\subsubsection{Justification of the Division of Labour}

The four major cells were divided between the team as recommended in the specification.
It was decided to also attempt the optional cells for teams of four.
An extra person in a team of five would be in charge of designing the halfadder and xor2.
The halfadder was assigned to the team member designing the fulladder as their expertise was greater for designing adder cells.
The xor2 cell was assigned to the team member designing the rdtype as this cell was assumed less time consuming than the two other sets of major cells.
The remaining cells were grouped and divided among the teams as shown in Table~\ref{tab:cells}.

\begin{table}[h]
   \centering
    \begin{tabular}{| p{4.5cm} | p{6.5cm} |}
    \hline
      \textbf{Cell Groupings} & \textbf{Reasoning}\\ \hline
      and2, and2, nand3, nand4  &	Nands and Ands are similar in design.\\ \hline
      buffer, trisbuf  &	Buffers are similar in design.\\ \hline
      inverter, nor2, nor3, or2   &	 Nors and Ors similar in design. Inverter is a simple cell assigned to distribute load.\\ \hline
      rowcrosswer, tiehigh, tielow &	Low complexity cells assigned to team member tasked with designing both adders.\\ \hline
    \end{tabular}
    \caption{Cell design groupings and reasoning}
    \label{tab:cells}
\end{table}

\subsubsection{Version Control}
The Git revision control system was used throughout the project to facilitate collaborative working.
GitHub is a web-based hosting service for git. 
This was used to share files, allocate work and track bugs.

\subsubsection{Process automation}
The team produced a number of scripts to improve productivity by automating processes such extraction from magic, design simulation and consistency checking.
Using \LaTeX allowed the databook compilation to also be automated as separate files for each cell were stored in the hierarchy of folders.   
This way each designer could add their own files to the data book independently.



\clearpage
\subsection{Division of Labour}


\clearpage
\section{Design Detail}
All HSpice simulations were done using a loaded cell.
All the inputs were driven by an inverter, and all outputs drove two inverters, seen in Figure \ref{fig:loadedsims}
Measurements were done between 50\% of the supply voltage between a change in the input to the cell, to a change in the output. 
Rise and fall times were measured for each signal and averaged. 

\begin{figure}[htb!]
\includegraphics[width=\textwidth]{loadedsims.pdf}
\caption{Block Diagram of the loaded simulations}
\label{fig:loadedsims}
\end{figure}

\renewcommand{\cellname}[1] {\newpage \subsection{#1} \makebox[\linewidth]{\rule{\textwidth}{0.4pt}}}
\newcommand{\stickdiagram}[1]{\subsubsection*{Stick Diagram} \begin{center}\includegraphics[width=\textwidth,height=10cm,keepaspectratio=true]{#1}\end{center}}
\newcommand{\transistor}[1]{\subsubsection*{Transistor Level Circuit Diagram} \begin{center}\includegraphics[width=\textwidth,height=8cm,keepaspectratio=true]{#1}\end{center}}
\newcommand{\spicesim}[1]{\subsubsection*{HSpice Simulation} \begin{center}\includegraphics[width=\textwidth]{#1}\end{center}}
\cleardoublepage
\cellname{fulladder}
\designer{Martin Wearn}
\celldescription{Adds two bit values and the previous bitslice carry out, to produce a sum and carry}

\stickdiagram{../fulladder/stickdiagram.pdf}
\transistor{../fulladder/transistorcd.pdf}

%TO DO : Results of loaded simulation bits
\cellname{HALFADDER}
\designer{Martin Wearn}
\celldescription{Adds two bits to produce a sum and carry}

\stickdiagram{../halfadder/stickdiagram.pdf}
\acchar{../halfadder/acresults.txt}
\transistor{../halfadder/transistorcd.pdf}
\spicesim{../halfadder/hspice.png}


\cellname{leftbuf}
\designer{Henry Lovett}
\celldescription{A start of row buffer cell.}

\stickdiagram{../leftbuf/stickdiagram.jpg}
\transistor{../leftbuf/transistorcd.jpg}

%TO DO : Results of loaded simulation bits
\cellname{rdtype}
\designer{Ashley Robinson}
\celldescription{Raw Dtype}

\stickdiagram{../rdtype/stickdiagram.jpg}
\transistor{../rdtype/transistorcd.jpg}

%TO DO : Results of loaded simulation bits
\cellname{smux2}
\designer{Schep}
\celldescription{A start of row buffer cell.}

\stickdiagram{../smux2/stickdiagram.pdf}
\transistor{../smux2/transistorcd.pdf}

%TO DO : Results of loaded simulation bits

\cellname{smux3}
\designer{Schep}
\celldescription{A scan multiplexor.}

\stickdiagram{../smux3/stickdiagram.jpg}
\transistor{../smux3/transistorcd.jpg}

%TO DO : Results of loaded simulation bits
\cellname{XOR2}
\designer{Ashley Robinson}
\celldescription{A 2 input XOR gate.}

Compound gate routed using an Euler path.

\stickdiagram{../xor2/stickdiagram.pdf}
\acchar{../xor2/acresults.txt}
\transistor{../xor2/transistorcd.pdf}
\spicesim{../xor2/hspiceColor.png}


\end{document}
%% ----------------------------------------------------------------
